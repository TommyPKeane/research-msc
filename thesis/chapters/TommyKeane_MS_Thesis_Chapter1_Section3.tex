%%%%%%%%%%%%%%%%%%%%%%%%%%%%%%%%%%%%%%%%%%%%%%%%%%%%%%%%%%%%%%%%%%%%%%%%%%%%%%%
%
% Tommy P. Keane
% Master of Science Thesis
% Department of Electrical and Microelectronic Engineering
%
% March 2011
%
%
%
% .tex and .sty modified from:
% http://www.ce.rit.edu/studentresources/gradresource/LaTexThesis.zip
%
%%%%%%%%%%%%%%%%%%%%%%%%%%%%%%%%%%%%%%%%%%%%%%%%%%%%%%%%%%%%%%%%%%%%%%%%%%%%%%%

%%%%%%%%%%%%%%%%%%%%%%%%%%%%%%%%%%%%%%%%%%%%%%%%%%%%%%%%%%%%%%%%%%%%%%%%%%%%%%%
%
% CHAPTER 1
%
% SECTION 3
%
%%%%%%%%%%%%%%%%%%%%%%%%%%%%%%%%%%%%%%%%%%%%%%%%%%%%%%%%%%%%%%%%%%%%%%%%%%%%%%%


%%%%%%%%%%%%%%%%%%%%%%%%%%%%%%%%%%%%%%%%%%%%%%%%%%%%%%%%%%%%%%%%%%%%%%%%%%%%%%%
% BEGIN DOCUMENT

The WFMI algorithm is a novel set of means to perform unsupervised, automated panorama creation for surveillance, and other realistic, scenes. The results of the algorithm are convincing, blended views generated by the affine homography derived from a feature correspondence based affine transform search between the scenes. This is a robust and novel affine registration algorithm that can provide useful estimations in registering views of scenes that would be more accurately registered by a projective homography, despite the lack of explicit projective scene relationship considerations in the algorithm itself. The decision to restrict the algorithm to an affine search space will be discussed in detail in Chapter 4. The initial implementation of an MMI algorithm gave light to the necessity of very interesting practical considerations for the application of mutual information as a robust correspondence metric. The novel weighting and filtering aspects of the WFMI algorithm go beyond a simple normalization process, these steps are actually crucial to getting accurately registered views from realistic scenes, especially in scenes with lower entropy or minimal amounts of overlap.

Given two frames from views with overlapping scene content, the algorithm presented here will perform completely unsupervised registration. There is no camera calibration step, there are no initial correspondences, and there is no camera location or orientation assumption. Any assumptions that were made will be detailed in Chapters 3 and 4. The required conditions for the scene are merely a modest level of scene entropy and an empirically derived minimum pixel overlap (in terms of the real-world scene correspondence). These will be discussed later on in more detail, but, again, these were empirically derived and are not hard limits. Note also that this algorithm is derived under affine and some near-affine constraints, though it succeeds in finding accurate registrations for near-affine related views and projective (or more complex) related views.

By developing a strong understanding of the principles of image registration, this algorithm shows that typical constraints and models can be expanded both mathematically and practically. Known camera parameters, camera spatial relationships, and \textit{a priori} point correspondences are shown to be significant aids to getting from two disparate images to a registered panoramic view, but the WFMI algorithm shows that these are not the only means to do so. Practical considerations motivated the generalization of the algorithm, which pushed the limits of contemporary research in automated image registration. The WFMI algorithm contributes a novel extension of mutual information based correspondence methods through application to complex scenarios with illumination variations, parallax, and occlusions.



%%%%%%%%%%%%%%%%%%%%%%%%%%%%%%%%%%%%%%%%%%%%%%%%%%%%%%%%%%%%%%%%%%%%%%%%%%%%%%%
% END OF DOCUMENT

