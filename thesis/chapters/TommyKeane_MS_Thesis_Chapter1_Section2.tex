%%%%%%%%%%%%%%%%%%%%%%%%%%%%%%%%%%%%%%%%%%%%%%%%%%%%%%%%%%%%%%%%%%%%%%%%%%%%%%%
%
% Tommy P. Keane
% Master of Science Thesis
% Department of Electrical and Microelectronic Engineering
% Rochester Institute of Technology
%
% April 2011
%
%
%
% Funded By: Lenel Systems Inc., A UTC Fire & Security Corporation
%
% Algorithm Intellectual Property Owned By: Lenel Systems Inc.
%
%
% http://www.tommypkeane.com
%
%%%%%%%%%%%%%%%%%%%%%%%%%%%%%%%%%%%%%%%%%%%%%%%%%%%%%%%%%%%%%%%%%%%%%%%%%%%%%%%

%%%%%%%%%%%%%%%%%%%%%%%%%%%%%%%%%%%%%%%%%%%%%%%%%%%%%%%%%%%%%%%%%%%%%%%%%%%%%%%
%
% CHAPTER 1
%
% SECTION 2: Literature Review
%
%%%%%%%%%%%%%%%%%%%%%%%%%%%%%%%%%%%%%%%%%%%%%%%%%%%%%%%%%%%%%%%%%%%%%%%%%%%%%%%


%%%%%%%%%%%%%%%%%%%%%%%%%%%%%%%%%%%%%%%%%%%%%%%%%%%%%%%%%%%%%%%%%%%%%%%%%%%%%%%
% BEGIN DOCUMENT

Contemporary work in image registration has progressed from basic applications of projective geometry theory to solving large-scale and complex problems. Research continues more and more in applications that extend the limits of the theoretical developments. The work presented here has looked at multiple views of complex scenes susceptible to parallax and object occlusion disparities, as well as the standard concerns of depth variation and the complexities of projective geometry within multiple views. Looking through the current research presents what has been done, what is developing in the field, and where and how the WFMI algorithm fits in with modern research.

 An excellent survey of the field of image registration was done by the Zitov{\'a} and Flusser in \cite{Zitova2003}. Their work very succinctly described the four major steps of most modern registration algorithms: feature detection, feature matching, transform calculation, and image transformation. This is an excellent distillation of all major techniques, and in Chapter 3 the WFMI algorithm's approach will be compared to this standard methodology. A lot of the research in the field is focused on the first two of the four steps; feature detection and matching. In these areas of research a new method appears rarely and most techniques are based on a few general methods that are refined as research progresses (cross-correlation, mutual information, Fourier methods, corners and edges, gradient descent algorithms, \etc). Foundational work that is still in very wide use today was done by \cite{Brown2003} and \cite{Brown2005} in looking for an affine invariant set of features and a method for their correspondence.
 
 There is also another well-written survey of medical image registration, and more specific to applying mutual information techniques \cite{Pluim2003}, from the same year (2003). Again, many of the techniques rely on only a few general algorithms that are then modified, aiming for refinement, with new measures or weighting schemes and various optimization techniques. The work in \cite{Viola1997} was foundational in mutual information techniques and provides a very in-depth understanding of the derivation of mutual information as a correspondence metric. Looking through \cite{Zitova2003} and \cite{Pluim2003} can give an excellent first exposure to image registration and the application of mutual information, but being over 8 years old at the time of this writing there are more contemporary techniques and research being applied.
 
 More recent research such as \cite{Walli2009}, \cite{Rav-Acha2005}, \cite{Nilosek2009}, \cite{Kanazawa2004}, \cite{Haenselmann2009}, \cite{Gracias2009}, \cite{Brown2007}, and \cite{Fan2008} have all advanced the field greatly by exploring new techniques and new applications for the general methods found in \cite{Zitova2003} and \cite{Pluim2003}. What is still of peak interest, though, is the feature detection and feature matching stages of general image registration. Homography calculation and image transformation are extremely well-theorized by the mathematics of projective geometry, but determining a homography through feature detection and feature matching seems to require more insight into the cognitive understanding of imagery.
 
  The work in \cite{Brown2003} is an excellent example of the application of David Lowe's foundational SIFT algorithm. SIFT stands for Scale Invariant Feature Transform, and is widely used in research for the feature detection and matching stages. Especially as research has progressed into recognizing projective geometry concerns and more complex multi-view scenarios, it is becoming more and more crucial to define invariant features, as opposed to the previous industry standard of using corner features as developed in \cite{Harris1988}. While Lowe's work does not provide the perfect invariant features, it is still the research standard, although it is a patented algorithm requiring a commercial license. Work such as \cite{Kanazawa2004} and \cite{Brown2007} are basic examples of the continued research on the SIFT operation. And while it is a great tool, the proprietary nature was incompatible with the grant-funded research done here. It is also a complex technique that only provides the features and estimated matches, it does not explicitly or uniquely correspond the features, often requiring the implementation of a RANSAC algorithm \cite{Brown2007}. This pushed our research away from a similar technique as it seemed unlikely to move towards an efficient and simple real-time implementation for color video surveillance scenarios.
  
  The work in \cite{Walli2009} and \cite{Nilosek2009} takes into consideration projective geometry, stereo image processing, and multi-view concerns in order to develop 3-D scene models. This was inspirational and motivational work for the WFMI algorithm as it presents the rich amount of data and information available in registered imagery. As will be discussed further in Chapter 5, the WFMI algorithm's success is applicable to future work in depth reconstruction, scene understanding, and the potential for becoming an iterative algorithm that could be self-improving in determining the precise view-to-view homography. Stereo correspondence provides an immense amount of information about scene content and geometries, making it a quickly emerging topic of interest.
  
  In terms of feature correspondence, RANSAC \cite{Brown2007} is still the most popular technique, while supervised correspondence continues to be in widespread use as well. The problem of feature correspondence is extremely difficult as it requires some knowledge or assumption of the structure of the views or the scenes once features have been identified. Once the features can be corresponded, the actual  registration of the images and the homography generation are extremely well defined by projective geometry as detailed extensively in \cite{Faugeras2004} and \cite{Hartley2003}. Leaving the only other major concern to be the actual stitching and display of the registered images.
  
   Again in \cite{Zitova2003} stitching techniques are described, and there are novel techniques still being developed as in \cite{Haenselmann2009}. The difficulty in stitching is that its accuracy is often entirely dependent upon the accuracy of the homography. Given that the WFMI algorithm can only assure an accurate estimate in general cases, most stitching techniques prove to be ill-posed as it is known that the transformed pixels are not perfectly corresponded in all results for the WFMI algorithm. However, the research done in \cite{Burt1983} developed a Laplacian pyramid blending technique that blends frequency content of images without requiring any spatial correspondence. In \cite{Burt1983} images are shown that have been blended for artistic effect, an apple and orange for example, which have not even been through a registration algorithm. Chapter 4 will discuss the results in detail, but the multi-resolution spline blending technique is vastly superior to any other color correction, stitching, or blending algorithm for this implementation because perfect correspondence accuracy is not required to create the convincing view. Ideally there should be a utilization of the technique in \cite{Haenselmann2009} combined with the work from \cite{Burt1983}, because then the location of the view transition (the stitching seam) could be hidden; yet since the WFMI algorithm is not providing complete accuracy in the general projective case it was determined as a matter of future research to improve the pixel-based correspondence accuracy of the panorama. Again this will be discussed in further detail in Chapter 5.
   
   The WFMI algorithm development took careful consideration of previous work, especially as highlighted by the research surveys in \cite{Zitova2003} and \cite{Pluim2003}. Yet the WFMI algorithm was tasked to overcome a very open problem: fully automatic registration, unknown camera locations, no camera calibration, and to produce a convincing panoramic view. These conditions are far more open-ended than most techniques in practice, and any work facing similar conditions were often solved through the use of SIFT and RANSAC, or their derivatives. Or there were the large-scale computation problems, such as the excellent investigation in \cite{Snavely2008} using public image databases for synthesized scene tourism. The WFMI algorithm has followed a different path from most of the modern research, but is deeply rooted in many shared theories, making it extremely important to understand the current work and their limitations and goals. The discussion presented here should also provide added insight into strengths and limitations of multi-view registration, and where the field of research has yet to extend.
 
 

%%%%%%%%%%%%%%%%%%%%%%%%%%%%%%%%%%%%%%%%%%%%%%%%%%%%%%%%%%%%%%%%%%%%%%%%%%%%%%%
% END OF DOCUMENT

