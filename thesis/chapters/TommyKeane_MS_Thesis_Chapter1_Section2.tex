%%%%%%%%%%%%%%%%%%%%%%%%%%%%%%%%%%%%%%%%%%%%%%%%%%%%%%%%%%%%%%%%%%%%%%%%%%%%%%%
%
% Tommy P. Keane
% Master of Science Thesis
% Department of Electrical and Microelectronic Engineering
%
% March 2011
%
%
%
% .tex and .sty modified from:
% http://www.ce.rit.edu/studentresources/gradresource/LaTexThesis.zip
%
%%%%%%%%%%%%%%%%%%%%%%%%%%%%%%%%%%%%%%%%%%%%%%%%%%%%%%%%%%%%%%%%%%%%%%%%%%%%%%%

%%%%%%%%%%%%%%%%%%%%%%%%%%%%%%%%%%%%%%%%%%%%%%%%%%%%%%%%%%%%%%%%%%%%%%%%%%%%%%%
%
% CHAPTER 1
%
% SECTION 2
%
%%%%%%%%%%%%%%%%%%%%%%%%%%%%%%%%%%%%%%%%%%%%%%%%%%%%%%%%%%%%%%%%%%%%%%%%%%%%%%%


%%%%%%%%%%%%%%%%%%%%%%%%%%%%%%%%%%%%%%%%%%%%%%%%%%%%%%%%%%%%%%%%%%%%%%%%%%%%%%%
% BEGIN DOCUMENT

Contemporary work in image registration has progressed from basic applications of projective geometry theory to solving large-scale and complex problems. Research continues more and more in applications that extend the limits of the theoretical developments. The work presented here has looked at views of complex scenes susceptible to parallax, that differs between the views, and object occlusions, as well as the standard concerns of depth and projective geometry with multiple views. Looking through the current research presents what has been done, what is developing in the field, and where and how the WFMI algorithm fits in with modern research.

 An excellent survey of the field of image registration was done by the Zitov{\'a} and Flusser in \ref{Zitova2003}. Their work very succinctly described the four major steps of most modern registration algorithms: feature detection, feature matching, transform calculation, and image transformation. This is an excellent distillation of all major techniques and in Chapter 3 the WFMI algorithm's approach will be compared to this standard methodology. A lot of the research in the field is focused on the first two of the four steps; feature detection and matching. In these areas of research a new method appears rarely and most techniques are based on a few general methods that are refined as research progresses (cross-correlation, mutual information, Fourier methods, corners and edges, gradient descent algorithms, \etc). Foundational work that is still in very wide use today was done by \ref{Brown2003} and \ref{Brown2005}.
 
 There is also another well-written survey of medical image registration, and more specifically applying mutual information techniques \ref{Pluim2003} from the same year, 2003. Again, many of the techniques rely on only a few general algorithms that are then modified, aiming for refinement, with new measures or weighting schemes and various optimization techniques. The work in \ref{Viola1997} was foundational in mutual information techniques and provides a very in-depth understanding of the derivation of mutual information as a correspondence metric. Looking through \ref{Zitova2003} and \ref{Pluim2003} can give an excellent first exposure to image registration and the application of mutual information, but being over 8 years old at the time of this writing there are more contemporary techniques and research being applied.
 
 More recent research such as \ref{Walli2009}, \ref{Rav-Acha2005}, \ref{Nilosek2009}, \ref{Yasushi2004}, \ref{Haenselmann2009}, \ref{Gracias2009}, \ref{Brown2007}, and \ref{Fan2008} have all advanced the field greatly by exploring new techniques and new applications for the general methods found in \ref{Zitova2003} and \ref{Pluim2003}. What is still of peak interest, though, is the feature detection and feature matching stages of general image registration.
 
  The work in \ref{Brown2003} is an excellent example of the application of David Lowe's foundational SIFT algorithm. SIFT stands for Scale Invariant Feature Transform, and is widely used in research for the feature detection stage. Especially as research has progressed into recognizing projective geometry concerns and more complex multi-view scenarios, it is becoming more and more crucial to define invariant features, as opposed to the previous industry standard of using corner features as developed in \ref{Harris1988}. While Lowe's work does not provide the perfect invariant features, it is still the research standard, although it is a patented algorithm requiring a commercial license. Work such as \ref{Yasushi2004} and \ref{Brown2007} are basic examples of the continued research on the SIFT operation. And while it is a great tool, the proprietary nature was incompatible with the grant-funded research done here. It is also a complex technique that only provides the features, it does not correspond the features (step 2), which is often done by implementation of a RANSAC algorithm \ref{Brown2007}. This pushed our research away from a similar technique as it seemed unlikely to move towards an efficient and simple real-time implementation for color video surveillance scenarios.
  
  The work in \ref{Walli2009} and \ref{Nilosek2009} takes into consideration projective geometry, stereo image processing, and multi-view concerns in order to develop 3-D scene models. This was inspirational and motivational work for the WFMI algorithm as it presents the possibilities available for the rich amount of data and information available in registered imagery. As will be discussed further in Chapter 6, the WFMI algorithm's success is applicable to future work in depth reconstruction, scene understanding, and the potential for becoming an iterative algorithm that could be self-improving in determining the proper view-to-view homography. Stereo correspondence provides an immense amount of information about scene content and geometries, making it a quickly emerging topic of interest.
  
  In terms of feature correspondence (step 2), RANSAC \ref{Brown2007} is still the most popular technique, while supervised correspondence is in widespread use as well. The problem of feature correspondence is extremely difficult as it requires some knowledge or assumption of the structure of the views or the scenes once features have been identified. Once the features can be corresponded the actual transformation and registration of the images, the homography generation, is extremely well defined by projective geometry as detailed extensively in \ref{Faugeras2004} and \ref{Hartley2003}. Leaving the only other major concern to be the actual stitching of the registered images.
  
   Again in \ref{Zitova2004} stitching techniques are described, and there are novel techniques still being developed as in \ref{Haenselmann2009}. The difficulty in stitching is that its accuracy is entirely dependent upon the accuracy of the homography. Given that the WFMI algorithm can only assure an accurate estimate in general cases, most stitching techniques prove to be ill-posed as it is known that transformed pixels are not perfectly corresponded. However, despite being 28 years old, the research done in \ref{Burt1983} developed a laplacian pyramid blending technique that blends frequency content of images without requiring any spatial correspondence. In \ref{Burt1983} images are shown that have been blended for artistic effect, an apple and orange for example, which have not even been through a registration algorithm. Chapter 5 will discuss the results, but the multiresolution spline blending technique is vastly superior to any other color correction, stitching, or blending algorithm as no correspondence accuracy is required. Ideally there should be a utilization of the technique in \ref{Haenselmann2009} with \ref{Burt1983}, as then the location of the view transition could be hidden, but as the WFMI algorithm is not providing complete accuracy in the general projective case it was determined as a matter of future research to improve the convincing nature of the panorama. Again this will be discussed in further detail in Chapter 6.
 
 



%%%%%%%%%%%%%%%%%%%%%%%%%%%%%%%%%%%%%%%%%%%%%%%%%%%%%%%%%%%%%%%%%%%%%%%%%%%%%%%
% END OF DOCUMENT

