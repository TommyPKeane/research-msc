%%%%%%%%%%%%%%%%%%%%%%%%%%%%%%%%%%%%%%%%%%%%%%%%%%%%%%%%%%%%%%%%%%%%%%%%%%%%%%%
%
% Tommy P. Keane
% Master of Science Thesis
% Department of Electrical and Microelectronic Engineering
%
% March 2011
%
%
%
% .tex and .sty modified from:
% http://www.ce.rit.edu/studentresources/gradresource/LaTexThesis.zip
%
%%%%%%%%%%%%%%%%%%%%%%%%%%%%%%%%%%%%%%%%%%%%%%%%%%%%%%%%%%%%%%%%%%%%%%%%%%%%%%%

%%%%%%%%%%%%%%%%%%%%%%%%%%%%%%%%%%%%%%%%%%%%%%%%%%%%%%%%%%%%%%%%%%%%%%%%%%%%%%%
%
% CHAPTER 6
%
% PREAMBLE
%
%%%%%%%%%%%%%%%%%%%%%%%%%%%%%%%%%%%%%%%%%%%%%%%%%%%%%%%%%%%%%%%%%%%%%%%%%%%%%%%


%%%%%%%%%%%%%%%%%%%%%%%%%%%%%%%%%%%%%%%%%%%%%%%%%%%%%%%%%%%%%%%%%%%%%%%%%%%%%%%
% BEGIN DOCUMENT

The presented algorithm generates convincing panoramic views automatically for complex scenes. The research here has investigated the robustness of mutual information as a metric for complex, realistic scenes. There is still a large amount of room for improvement, but this work has shown the strength and versatility of mutual information, especially for overcoming difficult problems such as parallax differences and object occlusions between multiple views. In the case of occlusion there is no perfect registration as portions of the scene are available in one view but not the other(s), and this is the case for the strength of the WFMI algorithm. Since there isn�t a reliance on pixel correspondences, but rather segment-type correspondence, occlusion is not a crippling problem for the WFMI registration algorithm. In the case of parallax differences between views, the appropriate homography may not exist and a Fundamental matrix or a non-linear polynomial mapping may be appropriate. Categorizing these scenarios as Projective in nature (as in: related by a projective homography) is an appropriate simplification, but the WFMI algorithm has shown that the even further simplification to the affine search-space can still produce accurate registration, or an appropriate estimate, in the face of occlusion and parallax complexities.
Future work for the algorithm has been proposed to investigate applying the robustness and computational efficiency of the algorithm to more complex and accurate registration algorithms. Especially in projective scenarios the resultant registration could be used as an estimate or intial conditions to limit a more robust feature-based search, such as current leading algorithms like SIFT and RANSAC. In terms of improving the algorithm itself, initial research showed promise in utilizing edge and corner features from segmentation maps of the frames. An automated, unsupervised registration algorithm, such as [XX], would be required to keep the algorithm automatic, but this could greatly increase computational complexity. The segmentation would need to be robust and be tuned more towards shapes rather than textures, as in typical scenarios texture information is often lost or distorted due to indoor or outdoor reflections or parallax disparities.

%%%%%%%%%%%%%%%%%%%%%%%%%%%%%%%%%%%%%%%%%%%%%%%%%%%%%%%%%%%%%%%%%%%%%%%%%%%%%%%
% END OF DOCUMENT
