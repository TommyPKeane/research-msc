%%%%%%%%%%%%%%%%%%%%%%%%%%%%%%%%%%%%%%%%%%%%%%%%%%%%%%%%%%%%%%%%%%%%%%%%%%%%%%%
%
% Tommy P. Keane
% Master of Science Thesis
% Department of Electrical and Microelectronic Engineering
% Rochester Institute of Technology
%
% April 2011
%
%
%
% Funded By: Lenel Systems Inc., A UTC Fire & Security Corporation
%
% Algorithm Intellectual Property Owned By: Lenel Systems Inc.
%
%
% http://www.tommypkeane.com
%
%%%%%%%%%%%%%%%%%%%%%%%%%%%%%%%%%%%%%%%%%%%%%%%%%%%%%%%%%%%%%%%%%%%%%%%%%%%%%%%

%%%%%%%%%%%%%%%%%%%%%%%%%%%%%%%%%%%%%%%%%%%%%%%%%%%%%%%%%%%%%%%%%%%%%%%%%%%%%%%
%
% CHAPTER 6
%
% SECTION 1
%
%%%%%%%%%%%%%%%%%%%%%%%%%%%%%%%%%%%%%%%%%%%%%%%%%%%%%%%%%%%%%%%%%%%%%%%%%%%%%%%


%%%%%%%%%%%%%%%%%%%%%%%%%%%%%%%%%%%%%%%%%%%%%%%%%%%%%%%%%%%%%%%%%%%%%%%%%%%%%%%
% BEGIN DOCUMENT

Originally this work was set out as a project to replace a system of manual image registration for surveillance videos by means of a fully automatic software implementation utilizing stationary cameras with unknown locations and no \textit{a priori} information, besides allowing for the assumption of some amount of overlap between views that can be registered. The complexity of the scenario and project guidelines went beyond the initial scope of the project, but the novel application of mutual information was introduced and tested. An initial expansion of the algorithm would be to finalize the registration for non-affine views by enhancing the algorithm with a pixel-to-pixel correspondence algorithm applied to the determined overlap region. Given that the WFMI algorithm provides intelligent estimates for the registration of views of complex scenes, relatively simple registration algorithms could be successful when applied to the estimated overlap region. In this respect the WFMI algorithm would be setting initial conditions to limit the search space for pixel-based registration, allowing for faster and more efficient computations in robust registration algorithms. In a practical implementation, the WFMI algorithm could provide an initial estimate on video feed frames and as more frames are calculated, refinements could be made to the initial estimate, especially as objects pass through the overlap regions of the views being registered.

There was also some initial research done in investigating the application of unsupervised image segmentation, such as the robust and accurate algorithm in \cite{Ugarriza2009}, to generate better features for object correspondence. Again, since the WFMI algorithm is a region-based registration, the more intelligently that features are generated to define object boundaries rather than intensity data, the more accurate the mutual information metric will be, as overlapping segments containing the same object will be a relatively unique statistical event sharing the same feature histograms. However, this could greatly increase computational complexity as the segmentation would need to be robust considering the practical scenarios: uncontrolled weather, illumination, and camera artifacts.

In terms of advancing the application of the weighted and filtered mutual information metric itself, there is a lot of contemporary research moving towards scene understanding and 3-dimensional (spatial) image and video data. Multiple views of a scene directly allow for the extension of the projective geometry to develop 3-D information about the structure of the scene, and the objects within it. If a scene's structure is unknown from its views and an accurate registration is not available, the WFMI metric could be applied to identify objects, regions, or even elements of the scene in motion that are correlated between the views. Object tracking, depth reconstruction, and motion estimation could all be rich areas of research for this novel application of information theory.

%%%%%%%%%%%%%%%%%%%%%%%%%%%%%%%%%%%%%%%%%%%%%%%%%%%%%%%%%%%%%%%%%%%%%%%%%%%%%%%
% END OF DOCUMENT
