%%%%%%%%%%%%%%%%%%%%%%%%%%%%%%%%%%%%%%%%%%%%%%%%%%%%%%%%%%%%%%%%%%%%%%%%%%%%%%%
%
% Tommy P. Keane
% Master of Science Thesis
% Department of Electrical and Microelectronic Engineering
% Rochester Institute of Technology
%
% April 2011
%
%
% Funded By: Lenel Systems Inc., A UTC Fire & Security Corporation
%
% Algorithm Intellectual Property Owned By: Lenel Systems Inc.
%
% http://www.tommypkeane.com
%
%%%%%%%%%%%%%%%%%%%%%%%%%%%%%%%%%%%%%%%%%%%%%%%%%%%%%%%%%%%%%%%%%%%%%%%%%%%%%%%

%%%%%%%%%%%%%%%%%%%%%%%%%%%%%%%%%%%%%%%%%%%%%%%%%%%%%%%%%%%%%%%%%%%%%%%%%%%%%%%
%
% CHAPTER 1
%
% SECTION 1: Objectives and Motivation
%
%%%%%%%%%%%%%%%%%%%%%%%%%%%%%%%%%%%%%%%%%%%%%%%%%%%%%%%%%%%%%%%%%%%%%%%%%%%%%%%


%%%%%%%%%%%%%%%%%%%%%%%%%%%%%%%%%%%%%%%%%%%%%%%%%%%%%%%%%%%%%%%%%%%%%%%%%%%%%%%
% BEGIN DOCUMENT
Image registration is such a rich topic of interest because in the currently available algorithms there are often more constraints than are practically allowable or desirable. For modern security surveillance scenes, one camera will never be enough to sufficiently monitor the regions of interest in a scene. It is also often a waste of time and money to have a technician or operator generate \textit{a priori} point correspondences, even though it would allow for extremely accurate registration, as will be discussed in the next section. Single-view time-lapsed video mosaicking provides very little practical purpose in the surveillance field since there is no temporal registration (only spatial) and the larger the scene the more temporal variation there will be across the data set. It is immediately clear that in order to identify, track, or monitor individuals or areas of interest, as is the goal of security and surveillance systems, sufficient coverage and temporally concurrent views are essential. Multi-view video registration would provide the two.

Contemporary panorama creation research is extensively studied in situations of large sets of input images with large amounts of redundant overlaps, and views are often taken from a central viewpoint or a single moving camera. There also advanced algorithms relying on the complex mathematics and numerical methods techniques in the development of registering sparse matrices. These are very important and interesting areas of research, but they are respectively too impractical or too complex for generating a simple, accurate, and automatic algorithm. Surveillance systems are designed with a focus on cost-benefit analysis, which tends towards utilizing fewer cameras with minimal overlap in order to cover the most surveyable area while maintaining continuous spatial and temporal observation. This is the main motivation for this algorithm, to look at a realistic scenario, the surveillance of complex real-world scenes, and provide an automatic, fast algorithm to stitch together overlapping views with no prior knowledge of camera relationships which will present a convincing view of the observed scene.

By understanding the desired scenarios, as will be discussed in Chapter 3, it was found that a standard MMI approach is significantly susceptible to false positives in realistic surveillance scenes. The goals of this research were to provide convincing views, fast and frugal processing, and a versatile and extensible algorithm. The academic and scientific pursuits of the completion of this thesis pushed for consistent theoretical development, while the motivation for the research was to meet the project goals and provide a useable algorithm. In order to succeed it was necessary to develop an in-depth understanding of the practical considerations that ultimately guided the applied theoretical development. The WFMI algorithm is a prime example of modern research on the implementation of maximized mutual information (MMI) based registration algorithms with a novel approach to the complexities that arise in realistic surveillance scenes.



%%%%%%%%%%%%%%%%%%%%%%%%%%%%%%%%%%%%%%%%%%%%%%%%%%%%%%%%%%%%%%%%%%%%%%%%%%%%%%%
% END OF DOCUMENT
