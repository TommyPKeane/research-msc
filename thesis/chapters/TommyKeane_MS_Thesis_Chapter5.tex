%%%%%%%%%%%%%%%%%%%%%%%%%%%%%%%%%%%%%%%%%%%%%%%%%%%%%%%%%%%%%%%%%%%%%%%%%%%%%%%
%
% Tommy P. Keane
% Master of Science Thesis
% Department of Electrical and Microelectronic Engineering
%
% March 2011
%
%
%
% .tex and .sty modified from:
% http://www.ce.rit.edu/studentresources/gradresource/LaTexThesis.zip
%
%%%%%%%%%%%%%%%%%%%%%%%%%%%%%%%%%%%%%%%%%%%%%%%%%%%%%%%%%%%%%%%%%%%%%%%%%%%%%%%

%%%%%%%%%%%%%%%%%%%%%%%%%%%%%%%%%%%%%%%%%%%%%%%%%%%%%%%%%%%%%%%%%%%%%%%%%%%%%%%
%
% CHAPTER 6
%
% PREAMBLE
%
%%%%%%%%%%%%%%%%%%%%%%%%%%%%%%%%%%%%%%%%%%%%%%%%%%%%%%%%%%%%%%%%%%%%%%%%%%%%%%%


%%%%%%%%%%%%%%%%%%%%%%%%%%%%%%%%%%%%%%%%%%%%%%%%%%%%%%%%%%%%%%%%%%%%%%%%%%%%%%%
% BEGIN DOCUMENT

The presented algorithm generates convincing panoramic views automatically for complex scenes. The research here has investigated the robustness of mutual information as a metric for complex, realistic scenes. There is still a large amount of room for improvement, but this work has shown the strength and versatility of mutual information, especially for overcoming difficult problems such as parallax differences and object occlusions between multiple views. In the case of occlusion it is important to keep in mind that there is no perfect registration as portions of the scene are available in one view but not in the other(s). In these cases, the best registration would be the one that can correspond the pixels of objects that exist in both views. This scenario is the prime case for the strength of the WFMI algorithm, since there is not a reliance on sparse pixel-to-pixel correspondences, but rather segment-to-segment correspondence. Therefore, occlusion is not a crippling problem for the WFMI registration algorithm. In the case of parallax differences between views and the complexities arising from the projective geometry of multi-view imaging, the appropriate homography may not exist and a Fundamental matrix or a non-linear polynomial mapping may be more appropriate. Categorizing these scenarios as Projective in nature (as in: related by a projective homography) was shown as an appropriate simplification in the manual registration examples, but the WFMI algorithm has shown that the even further simplification to the affine search-space can still produce accurate registration, or an appropriate estimate in the more complex scenarios.

%%%%%%%%%%%%%%%%%%%%%%%%%%%%%%%%%%%%%%%%%%%%%%%%%%%%%%%%%%%%%%%%%%%%%%%%%%%%%%%
% END OF DOCUMENT
