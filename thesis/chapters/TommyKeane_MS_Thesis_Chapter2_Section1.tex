%%%%%%%%%%%%%%%%%%%%%%%%%%%%%%%%%%%%%%%%%%%%%%%%%%%%%%%%%%%%%%%%%%%%%%%%%%%%%%%
%
% Tommy P. Keane
% Master of Science Thesis
% Department of Electrical and Microelectronic Engineering
% Rochester Institute of Technology
%
% April 2011
%
%
%
% Funded By: Lenel Systems Inc., A UTC Fire & Security Corporation
%
% Algorithm Intellectual Property Owned By: Lenel Systems Inc.
%
%
% http://www.tommypkeane.com
%
%%%%%%%%%%%%%%%%%%%%%%%%%%%%%%%%%%%%%%%%%%%%%%%%%%%%%%%%%%%%%%%%%%%%%%%%%%%%%%%

%%%%%%%%%%%%%%%%%%%%%%%%%%%%%%%%%%%%%%%%%%%%%%%%%%%%%%%%%%%%%%%%%%%%%%%%%%%%%%%
%
% CHAPTER 2
%
% SECTION 1: Symbol Conventions
%
%%%%%%%%%%%%%%%%%%%%%%%%%%%%%%%%%%%%%%%%%%%%%%%%%%%%%%%%%%%%%%%%%%%%%%%%%%%%%%%
\indent
Bolding will be used to describe vectors, and all vectors are assumed column vectors, while multi-dimensional arrays will not be bolded or labeled specially besides a general convention of a capital letter for their variable name. Array sizes will be denoted in Row-by-Column-by-Channel format with the typical convention of $\mathfrak{m} \times \mathfrak{n} \times \mathfrak{p}$ (Fraktur script symbols) and their respective index variables as $(i,j,k)$. Thus vectors will always be of length $\mathfrak{m}$, with index $i$ for the elements, and the elements will not be bolded, but will share the same symbol as the vector. Standard set theory symbols will be utilized, requiring special attention on bracket notations for intervals, such as $[a,b)$ meaning an interval from and including $a$, up to but not including $b$. Positive and negative infinity will never be considered to be in a closed interval. All other special notations and symbols will be explained when presented.

%%%%%%%%%%%%%%%%%%%%%%%%%%%%%%%%%%%%%%%%%%%%%%%%%%%%%%%%%%%%%%%%%%%%%%%%%%%%%%%
% BEGIN DOCUMENT


%%%%%%%%%%%%%%%%%%%%%%%%%%%%%%%%%%%%%%%%%%%%%%%%%%%%%%%%%%%%%%%%%%%%%%%%%%%%%%%
% END OF DOCUMENT
