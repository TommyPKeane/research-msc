%%%%%%%%%%%%%%%%%%%%%%%%%%%%%%%%%%%%%%%%%%%%%%%%%%%%%%%%%%%%%%%%%%%%%%%%%%%%%%%
%
% Tommy P. Keane
% Master of Science Thesis
% Department of Electrical and Microelectronic Engineering
% Rochester Institute of Technology
%
% April 2011
%
%
%
% Funded By: Lenel Systems Inc., A UTC Fire & Security Corporation
%
% Algorithm Intellectual Property Owned By: Lenel Systems Inc.
%
%
% http://www.tommypkeane.com
%
%%%%%%%%%%%%%%%%%%%%%%%%%%%%%%%%%%%%%%%%%%%%%%%%%%%%%%%%%%%%%%%%%%%%%%%%%%%%%%%

%%%%%%%%%%%%%%%%%%%%%%%%%%%%%%%%%%%%%%%%%%%%%%%%%%%%%%%%%%%%%%%%%%%%%%%%%%%%%%%
%
% CHAPTER 1
%
% SECTION 6: Thesis Outline
%
%%%%%%%%%%%%%%%%%%%%%%%%%%%%%%%%%%%%%%%%%%%%%%%%%%%%%%%%%%%%%%%%%%%%%%%%%%%%%%%


%%%%%%%%%%%%%%%%%%%%%%%%%%%%%%%%%%%%%%%%%%%%%%%%%%%%%%%%%%%%%%%%%%%%%%%%%%%%%%%
% BEGIN DOCUMENT

Chapter 2 will present the requisite background information. While some basics of digital image sampling, information theory, and probability theory will be assumed known. The second chapter is presented as the essential mathematical and conceptual theories required to understand the algorithm and its development. This includes a conceptual understanding of information theory and histogram generation, as well as a mathematical development focusing on probabilistic measures. Another major component is digital filtering. This area will be generally assumed to be known, but will focus more on the relevant application and conceptual understanding of filtering in digital imaging. The rest of the chapter will be made up of digital interpolation and decimation, image formation, digital sampling, digital color images, and digital pattern recognition theory. This chapter will push towards a higher level of understanding based on well-defined mathematics, as that was the crux of the success in the implementation of this algorithm. Chapter 2 will provide the theory needed and developed in taking the mathematical details and expanding them to the practical scenarios of the algorithm.

Within Chapter 3 will be the algorithm itself. As Chapter 2 is providing background mathematics, Chapter 3 is the application of the presented theories and concepts to describe the completed algorithm. Chapter 3 is heavily geared towards understanding the choices made in the inception and prototyping of the algorithm that lead to the current state of the algorithm. This information provides not only a very detailed understanding of what was done and achieved, but illuminates the necessary details required in judging the success of the algorithm.

Chapter 4 will discuss the implementation of the algorithm in full detail, specifically focusing on the development of the algorithm. While Chapters 2 and 3 will have already presented a rigorous mathematical understanding and the system level design of the algorithm, Chapter 4 will discuss this development in terms of the prototyping and testing stages. Many changes were made throughout the development of this algorithm, especially given that there was a substantial conceptual and mathematical divergence from standard MMI algorithms.

Chapter 5 will illuminate the results: how they were achieved, what they show, and what possibilities there are for further development to these examples. What will be presented in this chapter is not just a display of results, but also a discussion of what the algorithm produces and why. The conditions of the image capture and formation are the major factors in the variability of the results.

Finally, with Chapter 6, this paper looks back to what has been developed and forward to what extensions and applications apply. A summary of the algorithmic, scientific, practical, and academic results is presented. More detailed applications and possibilities are highlighted, given that the whole of the algorithm has since been presented. And ultimately a look towards the future is provided. Not just future work on the algorithm itself, but the future advancements in the field of image registration that could possibly branch off from what was found in this research.



%%%%%%%%%%%%%%%%%%%%%%%%%%%%%%%%%%%%%%%%%%%%%%%%%%%%%%%%%%%%%%%%%%%%%%%%%%%%%%%
% END OF DOCUMENT

